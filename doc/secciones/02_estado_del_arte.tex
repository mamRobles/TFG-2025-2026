\chapter{Estado del arte}
\section{Introducción}
%TODO redactar intro
contexto para hacer la revisión (en mi caso redactar una metodología de desarrollo de audiojuegos recopilando todos los avances para facilitar su uso), exponer los objetivos de la misma (ídem). Por qué es necesario y qué pretendo conseguir, metodología, qué preguntas tenía y cómo he generado la consulta, fuentes que he consultado, criterios de aceptación y exclusión…

\section{Revisión}
% TODO redactar revisión
 En este caso me gustaría clasificar tal vez en secciones según si hablamos de puramente acercamientos a los audiojuegos, audiojuegos mixtos (con visuales) o simplemente interfaces auditivas que ya existen para otras cosas que no son juegos pero se pueden aplicar.
 \subsection{Herramienta de gestión bibliográfica}
\subsection{Herramientas de desarrollo}
\subsection{Metodologías de desarrollo}
\section{Conclusión}
% TODO redactar conclusión
Conclusión: hallazgos de la revisión, puntos más fuertes o probados como resultado e introducir la metodología que se va a usar en la propuesta, para explicarla en el siguiente capítulo.
El software libre y sus licencias \cite{gplv3} ha permitido llevar a cabo una expansión del
aprendizaje de la informática sin precedentes.
