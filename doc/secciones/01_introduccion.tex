\chapter{Introducción}
\section{Contexto/Antecedentes}
%TODO Redactar Contexto
1. Qué audiojuegos o interfaces auditivas existen, en qué consiste mi TFG, explicar qué es exactamente un audiojuego, en qué dominio se sitúa, qué enfoques se vienen defendiendo, qué mercado hay para ellos, qué facilita su existencia…\\
2. Definiciones aclaratorias de términos que se vayan a usar más adelante o generalizaciones (por ejemplo que de ahora en adelante juego se refiera a audiojuego o algo así)\\
\section{Justificación/Motivación}
%TODO Redactar Motivación
Esto también es descripción del problema
    1. ¿Qué falta en el mercado?\\
    2. ¿Qué necesidades existen respecto a la accesibilidad auditiva en juegos?\\
    3. ¿Qué problemas tienen los que ya existen?\\
    4.  ¿Cómo se puede potenciar el desarrollo de audiojuegos?\\
    5.  ¿Por qué se puede potenciar?\\
    6. ¿Por qué merece la pena que yo trabaje en esto?\\
    7.  ¿Por qué debe hacer esa tarea necesariamente un ingeniero o ingeniera informáticos?\\
\section{Objetivos}
%TODO Redactar Objetivos
Objetivos generales numerados
menos concretos que se pueden descomponer en objetivos específicos. (puedo formular objetivos iniciales y luego objetivos modificados según las necesidades que surgen en el desarrollo)
\section{Estructura de la memoria}
%TODO Redactar Estructura de la memoria
Lo escribo al final
\section{Repositorio del trabajo}
%TODO Redactar Repositorio del trabajo
\section{Dónde se ha publicado}
%TODO Redactar Dónde se ha publicado
Probablemente un release de Github

Este proyecto es software libre, y está liberado con la licencia \cite{gplv3}.