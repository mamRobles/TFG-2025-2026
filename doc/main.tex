%%%%%%%%%%%%%%%%%%%%%%%%%%%%%%%%%%%%%%%%%
% Short Sectioned Assignment LaTeX Template Version 1.0 (5/5/12)
% This template has been downloaded from: http://www.LaTeXTemplates.com
% Original author:  Frits Wenneker (http://www.howtotex.com)
% License: CC BY-NC-SA 3.0 (http://creativecommons.org/licenses/by-nc-sa/3.0/)
%%%%%%%%%%%%%%%%%%%%%%%%%%%%%%%%%%%%%%%%%
% Este documento está basado en la plantilla de repositorio para trabajos de 
% fin de grado de la ESTIIT-UGR de Juan Julián Merelo Guervós
% https://github.com/JJ/plantilla-TFG-ETSIIT
%%%%%%%%%%%%%%%%%%%%%%%%%%%%%%%%%%%%%%%%%
% \documentclass[paper=a4, fontsize=11pt]{scrartcl} % A4 paper and 11pt font size
\documentclass[11pt, a4paper]{book}
\usepackage{geometry}
 \geometry{
 a4paper,
 left=25mm,
 right=25mm,
 top=25mm,
 bottom=25mm,
 }
\usepackage[T1]{fontenc} % Use 8-bit encoding that has 256 glyphs
\usepackage[utf8]{inputenc}
\usepackage{fourier} % Use the Adobe Utopia font for the document - comment this line to return to the LaTeX default
\usepackage{listings} % para insertar código con formato similar al editor
\usepackage[spanish, es-tabla]{babel} % Selecciona el español para palabras introducidas automáticamente, p.ej. "septiembre" en la fecha y especifica que se use la palabra Tabla en vez de Cuadro
\usepackage{url} % ,href} %para incluir URLs e hipervínculos dentro del texto (aunque hay que instalar href)
\usepackage{graphics,graphicx, float} %para incluir imágenes y colocarlas
\usepackage[gen]{eurosym} %para incluir el símbolo del euro
\usepackage{cite} %para incluir citas del archivo <nombre>.bib
\usepackage{enumerate}
\usepackage{hyperref}
\usepackage{graphicx}
\usepackage{tabularx}
\usepackage{booktabs}

\usepackage[table,xcdraw]{xcolor}
\hypersetup{
	colorlinks=true,	% false: boxed links; true: colored links
	linkcolor=black,	% color of internal links
	urlcolor=cyan		% color of external links
}
\renewcommand{\familydefault}{\sfdefault}
\usepackage{fancyhdr} % Custom headers and footers
\pagestyle{fancyplain} % Makes all pages in the document conform to the custom headers and footers
\fancyhead[L]{Grado en Ingeniería Informática } % Empty left header
\fancyhead[C]{} % Empty center header
\fancyhead[R]{María del Mar Martínez Robles} % My name
\fancyfoot[L]{Curso 2025-2026} % Empty left footer
\fancyfoot[C]{} % Empty center footer
\fancyfoot[R]{\thepage} % Page numbering for right footer
%\renewcommand{\headrulewidth}{0pt} % Remove header underlines
\renewcommand{\footrulewidth}{0pt} % Remove footer underlines
\setlength{\headheight}{13.6pt} % Customize the height of the header

\usepackage{titlesec, blindtext, color}
\definecolor{gray75}{gray}{0.75}
\newcommand{\hsp}{\hspace{20pt}}
\titleformat{\chapter}[hang]{\Huge\bfseries}{\thechapter\hsp\textcolor{gray75}{|}\hsp}{0pt}{\Huge\bfseries}
\setcounter{secnumdepth}{4}
\usepackage[Lenny]{fncychap}

\begin{document}

	% Plantilla portada UGR
	\input{portada/portada}

	% Plantilla prefacio UGR
	\thispagestyle{empty}

\begin{center}
{\large\bfseries Accesibilidad en videojuegos \\ para personas con visión reducida mediante estrategias de
sonificación }\\
\end{center}
\begin{center}
María del Mar Martínez Robles\\
\end{center}

%\vspace{0.7cm}

\vspace{0.5cm}
\noindent\textbf{Palabras clave}: \textit{software libre}
\vspace{0.7cm}

\noindent\textbf{Resumen}\\
	% TODO: Escribir resumen
1. Motivación del trabajo. (También hay una sección para eso más detallada en Introducción)\\
2. Objetivos iniciales. (También hay una sección para eso más detallada en Introducción) \\
3. La metodología, las tecnologías y herramientas utilizadas (luego se estudian en la sección de estado del arte y se explican en la propuesta) \\
4. Resultados y conclusiones a las que se llega. Más extensos en la sección de conclusiones y trabajos futuros \\
\cleardoublepage

\begin{center}
	{\large\bfseries Accessibility in video games
 \\ for people with low vision through
sonification strategies }\\
\end{center}
\begin{center}
	María del Mar Martínez Robles\\
\end{center}
\vspace{0.5cm}
\noindent\textbf{Keywords}: \textit{open source}, \textit{floss}
\vspace{0.7cm}

\noindent\textbf{Abstract}\\


\cleardoublepage

\thispagestyle{empty}

\noindent\rule[-1ex]{\textwidth}{2pt}\\[4.5ex]

D. \textbf{Antonio Bautista Bailón Morillas}, profesor del Departamento de Ciencias de la Computación e Inteligencia Artificial de la  E.T.S. de Ingenierías Informática y de Telecomunicación 

\vspace{0.5cm}

\textbf{Informo:}

\vspace{0.5cm}

Que el presente trabajo, titulado \textit{\textbf{Accesibilidad en videojuegos
para personas con visión reducida mediante estrategias de sonificación}},
ha sido realizado bajo mi supervisión por \textbf{María del Mar Martínez Robles}, y autorizo la defensa de dicho trabajo ante el tribunal
que corresponda.

\vspace{0.5cm}

Y para que conste, expiden y firman el presente informe en Granada a Junio de 2026.

\vspace{1cm}

\textbf{El/la director(a)/es: }

\vspace{5cm}

\noindent \textbf{Antonio Bautista Bailón Morillas}

\chapter*{Agradecimientos}






	% Índice de contenidos
	\newpage
	\tableofcontents

	% Índice de imágenes y tablas
	\newpage
	\listoffigures

	% Si hay suficientes se incluirá dicho índice
	\listoftables 
	\newpage

    % Indice de conceptos (en caso de que hubiera suficientes)
    \input{secciones/Glosario}
    \newpage
    
	% Introducción 
	\chapter{Introducción}
\section{Contexto/Antecedentes}
%TODO Redactar Contexto
1. Qué audiojuegos o interfaces auditivas existen, en qué consiste mi TFG, explicar qué es exactamente un audiojuego, en qué dominio se sitúa, qué enfoques se vienen defendiendo, qué mercado hay para ellos, qué facilita su existencia…\\
2. Definiciones aclaratorias de términos que se vayan a usar más adelante o generalizaciones (por ejemplo que de ahora en adelante juego se refiera a audiojuego o algo así)\\
\section{Justificación/Motivación}
%TODO Redactar Motivación
Esto también es descripción del problema
    1. ¿Qué falta en el mercado?\\
    2. ¿Qué necesidades existen respecto a la accesibilidad auditiva en juegos?\\
    3. ¿Qué problemas tienen los que ya existen?\\
    4.  ¿Cómo se puede potenciar el desarrollo de audiojuegos?\\
    5.  ¿Por qué se puede potenciar?\\
    6. ¿Por qué merece la pena que yo trabaje en esto?\\
    7.  ¿Por qué debe hacer esa tarea necesariamente un ingeniero o ingeniera informáticos?\\
\section{Objetivos}
%TODO Redactar Objetivos
Objetivos generales numerados
menos concretos que se pueden descomponer en objetivos específicos. (puedo formular objetivos iniciales y luego objetivos modificados según las necesidades que surgen en el desarrollo)
\section{Estructura de la memoria}
%TODO Redactar Estructura de la memoria
Lo escribo al final
\section{Repositorio del trabajo}
%TODO Redactar Repositorio del trabajo
\section{Dónde se ha publicado}
%TODO Redactar Dónde se ha publicado
Probablemente un release de Github

Este proyecto es software libre, y está liberado con la licencia \cite{gplv3}.

	% Descripción del problema y hasta donde se llega
	% Esto no hace falta en principio porque lo pongo en introducción

	% Estado del arte
	% 	1. Crítica al estado del arte
	% 	2. Propuesta
	\chapter{Estado del arte}
\section{Introducción}
%TODO redactar intro
contexto para hacer la revisión (en mi caso redactar una metodología de desarrollo de audiojuegos recopilando todos los avances para facilitar su uso), exponer los objetivos de la misma (ídem). Por qué es necesario y qué pretendo conseguir, metodología, qué preguntas tenía y cómo he generado la consulta, fuentes que he consultado, criterios de aceptación y exclusión…

\section{Revisión}
% TODO redactar revisión
 En este caso me gustaría clasificar tal vez en secciones según si hablamos de puramente acercamientos a los audiojuegos, audiojuegos mixtos (con visuales) o simplemente interfaces auditivas que ya existen para otras cosas que no son juegos pero se pueden aplicar.
 \subsection{Herramienta de gestión bibliográfica}
\subsection{Herramientas de desarrollo}
\subsection{Metodologías de desarrollo}
\section{Conclusión}
% TODO redactar conclusión
Conclusión: hallazgos de la revisión, puntos más fuertes o probados como resultado e introducir la metodología que se va a usar en la propuesta, para explicarla en el siguiente capítulo.
El software libre y sus licencias \cite{gplv3} ha permitido llevar a cabo una expansión del
aprendizaje de la informática sin precedentes.

	
	\chapter{Propuesta}

\section{Metodología de desarrollo utilizada}
% TODO redactar metodología
 en principio esto es la conclusión del estado del arte, ya oficializada

\section{Planificación temporal}
% TODO redactar planificación temporal
\section{Plan de gestión de riesgos}
% TODO redactar plan de gestión de riesgos
escueto pero presente, en relación a los retrasos de la estimación según el uso o cambio de las tencologías)
\section{Costes y presupuesto}
% TODO redactar costes y presupuesto
\section{Test y despliegue}
%TODO redactar test y despliegue
(qué test necesitaré?), probablemente de usabilidad, feedback de cómo de entendible o confuso es el juego, resultados de la implementación de las herramientas. También si es “jugable” sin ayuda externa.
\section{Desarrollo}
%TODO redactar desarrollo
Dentro de esta sección describimos en detalle la implementación de la metodología en un caso práctico, los problemas encontrados y elementos de visualización (diagramas, historias de usuario o lo que toque)
\subsection{Análisis}
% 1. Análisis de requisitos
	% 2. Análisis de las soluciones
	% 3. Solucion propuesta
	% 4. Análisis de seguridad
\subsection{Implementación}
\section{Pruebas finales de usuario}
%TODO redactar pruebas


	% Conclusiones
	\chapter{Conclusiones y trabajos futuros}
\section{Conclusiones}
%TODO redactar conclusiones
1. Resumen del objetivo general y problema abordado, valoración positiva o negativa sobre los resultados y una justificación de por qué\\
2. Repaso de los objetivos específicos indicando porcentaje de realización, resumen de lo que se ha hecho para conseguirlo e indicar dónde están las evidencias (capítulo o sección)\\
3. Qué del grado me sirvió para el TFG y qué tuve que aprender\\
4. Valoración personal sobre el TFG y resultados\\
\section{Trabajo futuro}
% TODO redactar trabajo futuro
tareas incompletas, nuevos objetivos e ideas, requisitos que no he tratado… se explican con un párrafo y cómo abordarlos.


	% Trabajos futuros


	
	\newpage
	\bibliography{bibliografia}
	\bibliographystyle{plain}
	
\end{document}
